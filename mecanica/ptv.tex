\section{Mecanica Analítica introdução}

Definiremos todos os conceitos utilizando o exemplo do pêndulo duplo abaixo.

\subsection{Graus de Liberdade}

Quantidade de coordenadas livres que descrevem o sistema por completo. \\Ou seja no exemplo acima, se sabemos o valor de $\theta_1$ e o valor ed $\theta_2$ sabemos a posição de todos os elementos do pêndulo.

\subsection{Vínculo}

Restrições ao movimento de um corpo.\\
No caso do exemplo, temos dois vínculos, a \textbf{articulação da barra 1} ao teto e a \textbf{articulação da barra 2} que liga ela a barra 1.\\

O vínculos possuem \textbf{equações vínculares}, que são a descrição algébrica da restrição imposta pelo messmo.

No exemplo do pêndulo, caso adotemos as coordenadas do sistema como $(x_1, y_1, x_2, y_2)$, teremos que as equações vinculares seriam:

$\begin{cases} (x_1 - a)^2 + (y_1 - b)^2 = L^2 \\ (x_2 - x_1)^2 + (y_2 - y_1)^2 = L^2 \end{cases}$

No caso teriamos 4 coordenadas, porém 2 equações vinculares, resultando em 2 graus de liberdade.

\begin{namedtheorem}[Vínculo Holônomo Reônomo]
  Vínculos que possuem dependencia do tempo em suas equações.
\end{namedtheorem}

\begin{namedtheorem}[Vínculos Holônomo Esclerônomo]
  Vínculos que não possuem dependência do tempo em suas equações.
\end{namedtheorem}

\begin{namedtheorem}[Vínculos Não Holônomo]
  Vínculos que podem ser expressos apenas em equações diferenciais, cuja integração analítica é impossível.
  Neste tipo de sisema são necessárias mais coordenadas para a descrição da \textit{configuração} do que o número de graus de liberdade, visto que eles não possibilitam eliminar variáveis como equações de \textit{vínculos holônomos}.
\end{namedtheorem}

\subsection{Configuração}

No exemplo acima visualizamos que é possível descrever o mesmo sistema de diversas manerias, poderiamos \\\textbf{(1)} utilizar quatro coordendas $(x_1, y_1, x_2, y_2)$ \\\textbf{(2)} utilizar duas coordenadas $(\theta_1, \theta_2)$

Lançamos mão do conceito de \textit{coordenadas generalizadas}, são as coordenadas independentes do sistema em um determinado sistema, ou seja $(\theta_1, \theta_2)$ seriam coordenadas generalizadas.

E a \textit{configuração} do sistema é equivalente à posição dada pelas \textit{coordenadas generelizadas}, e o conjunto de coordenadas generalizadas define o \textit{espaço da configuração}.

\begin{namedtheorem}[Válidade do espaço de configuração]
Dado uma configuração tal que: \\
$q_1, q_2, \cdots, q_n$ são as coordenadas generalizadas\\
$m$ número de equações vinculares no espaço de coordenadas generalizadas\\
$x_1, x_2, \cdots, x_j$ coordenadas no sistema cartesiano\\
$l$ numero de equações vínculares no sistema cartesiano \\
$f_1, f_2, \cdots, f_j$ funções de transformação de coordenadas generalizadas para o cartesiano\\

Teremos que a configuração é representativa se:

\begin{enumerate}
	\item $GL = n - m = j - l$
	\item $J = |\frac{\partial (f_1, f_2, \cdots, f_j)}{\partial (q_1, q_2, \cdots, q_n)}| \neq 0$
\end{enumerate}

\end{namedtheorem}











