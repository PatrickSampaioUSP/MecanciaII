\section{Impulso}

As hipóteses assumidas no desenvolvimento da teoria são:

\begin{enumerate}
	\item Enquanto ocorre o \textbf{Impulso} não há variação de posição
	\item Devido o \textbf{Impulso} há variação de velocidade instantânea
	\item Duranto o \textbf{Impulso}, o efeito de outras forças impulsivas são desprezíveis
\end{enumerate}

\begin{namedtheorem}[Definição de Impulso]
$$ \norm{\vec{I}} = \int_{0}^{\delta t} \vec{F}\cdot dt $$
\end{namedtheorem}

\subsection{Teorema da Resultante do Impulso - TRI}

Hipóteses:

\begin{enumerate}
	\item $\vec{r}_i = \vec{r'}_i$
	\item $\vec{\dot{r}}_i \neq \vec{\dot{r'}}_i$
\end{enumerate}

Tem-se que a definição do TRI é:

$$ \boxed{I = m\delta V_g} $$

\subsection{Teorema do Momento de Impulso - TMI}

Definição para partículas

$$\sum_{i=1}^N \vec{r_i}\wedge m_i(\vec{V_{p_i}'} - \vec{V_{p_i}}) = \vec{M_o}$$

Para um corpo rígido

$$m(G-O)\wedge \delta \vec{v_o} + [i,j,k]J_o[\delta \omega] = \vec{M_o}$$

Havendo as classicas simplificações

\begin{enumerate}
	\item $O=G$, quando o polo escolhido for correspondente ao centro de massa
	\item Quando $O$ for um ponto fixo
\end{enumerate}

$$ \boxed{M_o = [i,j,k]J_o\delta\omega_o}  $$

As vezes pode ser interessante na resolução de exercícios considerar o \textit{momento angular} de cada corpo ou considerar o \textit{momento angular total do sistema}. Nesta última opção, o impulso entre corpos é desconsiderado, avalia-se apenas \textit{impulsos externos ao sistema}.

\subsection{Centro de Percussão}

Em algumas situações existem pontos de um corpo em que se forem aplicados \textit{forças impulsivas}, não irá ocorrer uma resposta impulsiva de eventuais articulações deste sistemas. A estes pontos, chamamos de \textbf{Centro de Percussão}. É importante salientar que a \textbf{condição necessária} para que haja um centro de percussão é a de que o corpo esteja rotacionando em relação a um de seus \textbf{eixos principais }, para que os \textbf{produtos de inercia sejam 0}.


