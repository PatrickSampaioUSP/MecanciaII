\section{Equações Lagrange}

Considerando deslocamentos compátiveis com os vínculos em um corpo rígido

\subsection{Forças Generalizadas}

O trabalho virtual sendo definido por:

$\delta \tau = (F_1\frac{\partial x_1}{\partial q_1} + F_1\frac{\partial x_2}{\partial q_1} + F_3\frac{\partial x_3}{\partial q_1})\delta q_1 + (F_1\frac{\partial x_1}{\partial q_2} + F_1\frac{\partial x_2}{\partial q_2} + F_3\frac{\partial x_3}{\partial q_2})\delta q_2$

Teremos as \textit{forças generalizadas} são definidas como:

$$Q_1 \coloneqq (F_1\frac{\partial x_1}{\partial q_1} + F_1\frac{\partial x_2}{\partial q_1} + F_3\frac{\partial x_3}{\partial q_1})$$
$$Q_2 \coloneqq (F_1\frac{\partial x_1}{\partial q_2} + F_1\frac{\partial x_2}{\partial q_2} + F_3\frac{\partial x_3}{\partial q_2})$$

Conceitualmente, a força generaliza \textbf{não é uma força}, mas sim uma quantidade que multiplicada por um deslocamento em uma coordenada generalizada fornece o valor do trabalho que uma força ativa produziria naquela coordenada. Apesar de não serem uma força, possuem unidade de força para que multiplicada por um deslocamento forneça unidade de energia.

\subsection{Equações de Lagrange}

$$ \boxed{\frac{d}{dt}(\frac{\partial T}{\partial \dot{q_1}}) - \frac{\partial T}{\partial q_1} = Q_1}$$
$$ \boxed{\frac{d}{dt}(\frac{\partial T}{\partial \dot{q_2}}) - \frac{\partial T}{\partial q_2} = Q_2}$$

\subsection{Trabalho de Potencial conservativo}

O trabalho realizado por uma força conservativa de uma configuração do sistema $P$ até à referência do sistema $P_0$ é definido como a energia potêncial.

$$ V = \int_{P}^{P_0} \sum_{P}^{P_0} $$

$$ V = - \int_{P_0}^{P} \sum_{P}^{P_0} $$

$$ \boxed{\delta \tau = (-\frac{\partial V_1}{\partial x_1})\delta x_1 + (-\frac{\partial V_2}{\partial x_2})\delta x_2 + (-\frac{\partial V_3}{\partial x_3})\delta x_3}$$

\subsection{Equações de Lagrange com função potencial}

$$Q_j = \sum_{i=1}^{3N} (-\frac{\partial V}{\partial x_i}\frac{\partial x_i}{\partial q_j}) = -\frac{\partial V}{\partial q_j} $$

$$ \frac{d}{dt}(\frac{\partial T}{\partial \dot{q_j}}) - \frac{\partial T}{\partial q_j} = Q_j $$

$$ \frac{d}{dt}(\frac{\partial T}{\partial \dot{q_j}}) - \frac{\partial T}{\partial q_j} + \frac{\partial V}{\partial q_j} = 0 $$

$$ \boxed{\frac{d}{dt}(\frac{\partial T}{\partial \dot{q_j}}) - \frac{\partial T - \partial V}{\partial q_j} = 0} $$

\subsection{Força dissipativa}

$$ R = \frac{1}{2} \sum_{i=1}^{3N} c_k\dot{x_i}^2, k=1, \dots N $$

\subsection{Langrange mais geral}

$$ \boxed{\frac{d}{dt}(\frac{\partial T}{\partial \dot{q_j}}) - \frac{\partial T - \partial V}{\partial q_j} + \frac{\partial R}{\partial \dot{q_j}} = Q_{NC}} $$






