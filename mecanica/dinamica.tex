\section{Dinâmica}

Conforme visto na seção passada, é possível utilizar os conceitos de \textit{Mecânica Analítica} para resolver problemas de \textbf{estática}, porém agora iremos introduzir o \textit{Princípio de D'Alembert} que permite a resolução de problemas de dinâmica com o mesmo ferramental.

\subsection{Princípio de D'Alambert}

Simplesmente definido como:\\

Das Leis de newton temos que:
$$ \vec{F} = m\vec{a}$$
$$ \vec{F} - m\vec{a} = 0$$

Define-se \textit{Força de Inércia} como $F_I = -m\vec{a}$ \\
$$m\vec{a} = \vec{F_I}$$

\textbf{Principio de D'Alambert}:

\begin{equation}
	\boxed{\vec{F} - F_I = 0}
\end{equation}

A \textit{Força de Inércia} não contribui no movimento da partícula, é chamada de uma força fictícia.

\subsection{Princípio de D'Alambert em um sistema de pontos}

Haverão três forças em jogo:

\begin{enumerate}
	\item Forças externas ao sistema, $\vec{F_{i}}$
	\item Forças de interação entre as partículas, $\vec{f_{ij}}$
	\item Forças de inércia de cada partícula, $F_{I_i}$
\end{enumerate}

Para uma partícula $i$ temos que:

$$ \vec{F_i} + \vec{f_{ij}} + \vec{F_{I_i}} = \vec{0} $$

Para um sistema de particular:

$$ \sum_{i=0}^N\sum_{j=0}^N f_{ij} = 0$$

E da definição do \textit{Princípio de D'Alambert}

\begin{equation}
\sum_{i=0}^N \vec{F_{i}} + \vec{f_{ij}} = 0
\label{principio_d_alambert}
\end{equation}


Portanto enuncia-se o teorema como:\\
\textit{Um sistema de partículas sempre possuirá equilibrio entre suas forças de inercia e as forças externas atuantes neste sistema.}

\subsection{Principio de D'Alambert e o PTV}

As forças externas serão divididas em $\vec{F_i}^a$ que são \textit{forças externas ativas} e $\vec{F_i}^v$ que são \textit{forças externas vinculares}.

Teremos que na situação elucidade na equação \ref{principio_d_alambert}, podemos estabelecer uma situação de "Equilibrio" quando a somatória das forças reais e ficticias são 0, logo podemos utilizar o PTV, com a ideia do trabalho virtual.

\begin{equation}
		\delta \tau = \sum_{i=0}^N (\vec{F_i}^a + \vec{F_i}^v - m_1\ddot{\vec{r_i}})\delta \vec{r_i} = 0
\end{equation}

Por fim teremos que pelo fato do movimento respeitar as equações vinculares:

\begin{equation}
	\sum_{i=0}^N (\vec{F_i}^v = 0
\end{equation}

Logo a forma \textit{lagrangeana} para o \textit{princípio de D'Alambert} é:

\begin{equation}
		\delta \tau = \sum_{i=0}^N (\vec{F_i}^a - m_1\ddot{\vec{r_i}})\delta \vec{r_i} = 0
\end{equation}

