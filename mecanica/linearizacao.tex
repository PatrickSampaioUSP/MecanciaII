\section{Linearização}

As simplificações mais usuais ocorrem nos sistemas que são dos seguintes tipos

\begin{enumerate}
	\item Energia Cinética é dada apenas pelo termo quadrático $T_2$, independe explicitamente do tempo
	\item Sitema Conservativo, foras generalizadas são decorrentes de um potêncial
	\item Vínculos são todos holônomos, portanto a config é dada pelas n coordenadas generalizadas
\end{enumerate}

A energia cinética de um sistema pode ser dividida em três termos

$$ T(q, \dot{q}, t) = T_2 + T_1 + T_0 $$

\begin{namedtheorem}[T_2]

$$ T_2 = \frac{1}{2}\sum_{j=1}^n \sum_{r=1}^n [\sum_{i=1}^{3n} m_k(\frac{\partial x_i}{\partial q_j} \frac{\partial x_i}{\partial q_r})] \dot{q_j} \dot{q_r} $$

Onde define-se

$$ a_{jr} = \sum_{i=1}^{3N} m_k(\frac{\partial x_i}{\partial q_j} \frac{\partial x_i}{\partial q_r}) $$

$$T_2 = \frac{1}{2}\sum_{j=1}^n \sum_{r=1}^n a_{jr}  $$

\end{namedtheorem}

\begin{namedtheorem}[T_1 Centrífugo]
$$ T_1 = \sum_{j=1}^n a_j\dot{q_j} $$

Onde

$$ a_j = \sum_{i=1}^{3N} m_k (\frac{\partial x_i}{\partial q_j} \frac{\partial x_i}{\partial t}) $$

\end{namedtheorem}

\begin{namedtheorem}[T_0 Giroscópico]

$$ T_0 = \frac{1}{2} \sum_{i=1}^{3N} m_k (\frac{\partial x_i}{\partial t})^2 $$

\end{namedtheorem}

A linearização é a aproximação por taylor dos termos em torno de um ponto de equilibrio. Através disto é possível obter equações lineares que vão aproximar o comportamento do sistema em deslocamentos pequenos em relação à este ponto.

$$ \boxed{M_q \ddot{q} + D\dot{u} + K u \approx 0} $$

\begin{namedtheorem}[Matriz da energia cinética]

$$ \boxed{M(q) = [\frac{\partial^2 T}{\partial \dot{q}^2}] = [\frac{\partial^2 T}{\partial \dot{q_i} \partial \dot{q_j}}]} $$

Propriedades:

\begin{itemize}
	\item Matriz simétrica por construção
	\item Matriz de massa
	\item Deve obdecer critérios de ponto de mínimo de uma matriz de derivadas parciais(Matriz hessiana)
	$$ det K > 0 $$
	\item Existem casos em que esta matriz pode ser positiva semi-definida, exigindo tratamento especial
\end{itemize}

\end{namedtheorem}

\begin{namedtheorem}[Matriz da energia potencial]

$$ \boxed{K = [\frac{\partial^2 V}{\partial q^2}] = [\frac{\partial^2 V}{\partial q_i \partial q_j}]} $$

Propriedades:

\begin{itemize}
	\item Matriz simétrica por construção
	\item Matriz de rigidez
	\item Deve obdecer critérios de ponto de mínimo de uma matriz de derivadas parciais(Matriz hessiana)
	$$ det K > 0 $$
\end{itemize}

\end{namedtheorem}

\begin{namedtheorem}[Matriz de amortecimento]

$$ \boxed{D = [\frac{\partial^2 R}{\partial q^2}] = [\frac{\partial^2 R}{\partial q_i \partial q_j}]} $$

Propriedades:

\begin{itemize}
	\item Matriz simétrica por construção
	\item Matriz de Amortecimento
	\item Deve obdecer critérios de ponto de mínimo de uma matriz de derivadas parciais(Matriz hessiana)
	$$ det K > 0 $$
\end{itemize}

\end{namedtheorem}

\begin{namedtheorem}[Condições de Equilibrio]

$$ \boxed{\frac{\partial V}{\partial q_i} = 0} $$

\begin{enumerate}
	\item Se a matriz hessiana da energia potencial for positiva definida, o ponto será de equilibrio estável
	\item Se a matriz hessiana for positiva semi-definida, o ponto será de cela
	\item Se a matriz hessiana for negativa, o equilibrio será instavel no ponto de estudo
\end{enumerate}

\end{namedtheorem}