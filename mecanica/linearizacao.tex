\section{Linearização}

As simplificações mais usuais ocorrem nos sistemas que são dos seguintes tipos

\begin{enumerate}
	\item Energia Cinética é dada apenas pelo termo quadrático $T_2$, independe explicitamente do tempo
	\item Sitema Conservativo, foras generalizadas são decorrentes de um potêncial
	\item Vínculos são todos holônomos, portanto a config é dada pelas n coordenadas generalizadas
\end{enumerate}

A energia cinética de um sistema pode ser dividida em três termos

$$ T(q, \dot{q}, t) = T_2 + T_1 + T_0 $$

\begin{namedtheorem}[T_2]

$$ T_2 = \frac{1}{2}\sum_{j=1}^n \sum_{r=1}^n [\sum_{i=1}^{3n} m_k(\frac{\partial x_i}{\partial q_j} \frac{\partial x_i}{\partial q_r})] \dot{q_j} \dot{q_r} $$

Onde define-se

$$ a_{jr} = \sum_{i=1}^{3N} m_k(\frac{\partial x_i}{\partial q_j} \frac{\partial x_i}{\partial q_r}) $$

$$T_2 = \frac{1}{2}\sum_{j=1}^n \sum_{r=1}^n a_{jr}  $$

\end{namedtheorem}

\begin{namedtheorem}[T_1 Centrífugo]
$$ T_1 = \sum_{j=1}^n a_j\dot{q_j} $$

Onde

$$ a_j = \sum_{i=1}^{3N} m_k (\frac{\partial x_i}{\partial q_j} \frac{\partial x_i}{\partial t}) $$

\end{namedtheorem}

\begin{namedtheorem}[T_0 Giroscópico]

$$ T_0 = \frac{1}{2} \sum_{i=1}^{3N} m_k (\frac{\partial x_i}{\partial t})^2 $$

\end{namedtheorem}